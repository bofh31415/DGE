\documentclass[a4paper, 11pt]{article}

% === PACKAGES ===
\usepackage[utf8]{inputenc}
\usepackage[T1]{fontenc}
\usepackage[ngerman]{babel}
\usepackage{geometry}
\usepackage{amsmath, amssymb}
\usepackage{tikz}
\usetikzlibrary{matrix,positioning,arrows.meta,calc,shapes.geometric,patterns}
\usepackage{enumitem}
\usepackage{fancyhdr}
\usepackage{array}

\geometry{left=2.5cm, right=2.5cm, top=2.5cm, bottom=2.5cm}

% === HEADER ===
\pagestyle{fancy}
\fancyhf{}
\rhead{Patentanmeldung -- DGE (Technische Speichersteuerung)}
\lhead{Sven Jansen, Dipl.-Inf.}
\cfoot{\thepage}

\begin{document}

% ============================================================================
% DECKBLATT
% ============================================================================
\begin{center}
\Large\textbf{PATENTANMELDUNG}\\[0.5cm]
\large beim Deutschen Patent- und Markenamt (DPMA)\\[1cm]
\end{center}

\noindent\textbf{Bezeichnung der Erfindung:}\\
\textit{Computerimplementiertes Verfahren und Vorrichtung zur speichereffizienten Steuerung von Schreibzugriffen auf Matrixstrukturen in adaptiven Signalverarbeitungssystemen mittels kontextabhängiger Signalgatterung}\\[0.5cm]

\noindent\textbf{Anmelder:}\\
Sven Jansen, Dipl.-Inf.\\
[Adresse]\\[0.5cm]

\noindent\textbf{Erfinder:}\\
Sven Jansen, Dipl.-Inf.\\[0.5cm]

\noindent\textbf{Technisches Gebiet:}\\
Die Erfindung betrifft allgemein die digitale Datenverarbeitung und insbesondere die ressourceneffiziente Verwaltung von Speicherstrukturen in adaptiven Systemen (wie z.B. künstlichen neuronalen Netzen) auf Hardware mit begrenzten Ressourcen (Embedded Systems, Echtzeit-Prozessoren).\\[1cm]

\hrule
\vspace{1cm}

% ============================================================================
% ZUSAMMENFASSUNG
% ============================================================================
\section*{Zusammenfassung}

Die Erfindung betrifft ein Verfahren zur technischen Steuerung von Speicherupdates in einem adaptiven Signalverarbeitungssystem. Das technische Problem besteht darin, die Funktionsfähigkeit eines bestehenden Systems für neue Eingangssignale zu erweitern, ohne den Speicherbedarf zu verdoppeln oder die Verarbeitungsgeschwindigkeit durch redundante Systeme zu beeinträchtigen. 

Die Lösung besteht in einer \textbf{kontextabhängigen Schreibsperre} (Gating), die auf der Hardware-Ebene den Signalfluss steuert. Ein Kontext-Detektor (Bigram-Router) analysiert die zeitliche Abfolge von Eingangssignalen $t-1 \to t$, um den aktuellen Betriebsmodus zu bestimmen. Basierend darauf wird ein Steuersignal generiert, das selektiv den Schreibzugriff (Gradienten-Update) auf spezifische physikalische Speicheradressen blockiert ("Frozen Core"), während andere Adressen für neue Daten freigegeben werden ("Expansion Area"). Dies ermöglicht die \textbf{Mehrfachnutzung physischer Ressourcen} ohne Interferenz (Datenkollision), wodurch die Hardware-Effizienz signifikant gesteigert wird.

\newpage

% ============================================================================
% BESCHREIBUNG
% ============================================================================
\section{Technischer Hintergrund}

\subsection{Stand der Technik und Nachteile}

In modernen adaptiven Computerarchitekturen, insbesondere bei der Verarbeitung sequenzieller Daten (z.B. Sensordatenströme, Textverarbeitung), werden Parameter in großen Matrixstrukturen im Speicher (RAM/VRAM) abgelegt.
Soll ein solches System an neue Anforderungsprofile (neue Aufgaben) angepasst werden, existieren im Stand der Technik zwei ineffiziente Ansätze:
\begin{enumerate}
    \item \textbf{Vollständiges Nachtraining:} Alle Speicherwerte werden überschrieben. Dies führt technisch zum Verlust der vorherigen Konfiguration ("Katastrophales Vergessen"). Das System ist nicht mehr abwärtskompatibel.
    \item \textbf{System-Duplizierung:} Für jede neue Aufgabe wird eine Kopie des Systems angelegt. Dies führt zu einem linearen Anstieg des Speicherbedarfs ($O(N)$), was auf eingebetteten Systemen (Automotive, Mobile) physikalisch oft nicht realisierbar ist.
\end{enumerate}

\subsection{Das technische Problem}
Der Erfindung liegt die technische Aufgabe zugrunde, ein Verfahren bereitzustellen, das es ermöglicht, die Funktionalität eines adaptiven Systems zu erweitern, \textbf{ohne} den physikalischen Speicherbedarf linear zu skalieren und \textbf{ohne} die Integrität bestehender Datenkonfigurationen zu gefährden.

\section{Technische Lösung: Kontextbasierte Speichersteuerung}

Die Aufgabe wird durch eine Vorrichtung gelöst, die eine dynamische Segmentierung des physikalischen Speichers vornimmt. 

\subsection{1. Signalkontext-Erkennung (Technischer Trigger)}

Herkömmliche Steuersysteme (MLP) betrachten nur den isolierten Eingangswert $x_t$. Dies führt bei identischen Werten in unterschiedlichen Prozessphasen zu Fehlsteuerungen (Aliasing).
Die erfindungsgemäße Vorrichtung umfasst einen \textbf{Kontext-Detektor}, der ein Zustandssignal aus der Kombination des aktuellen Werts $x_t$ und eines gepufferten Vorwerts $x_{t-1}$ bildet.
Dieses zusammengesetzte Signal ([Signal A; Signal B]) dient als technischer Trigger ("Schlüssel"), um eindeutig zwischen Betriebsmodi (z.B. "Alter Prozess" vs. "Neuer Prozess") zu unterscheiden, selbst wenn die Einzelwerte identisch sind.

\subsection{2. Asymmetrische Schreibsteuerung (Resource Locking)}

Basierend auf dem Triggersignal steuert die Vorrichtung zwei Hardware-Komponenten an:
\begin{itemize}
    \item \textbf{Adress-Selektor (Router):} Leitet den Datenstrom auf einen spezifischen Sub-Bereich der Speichermatrix.
    \item \textbf{Schreib-Controller (Gradient Blocker):} Wendet während einer Anpassungsphase (Training) eine selektive Schreibsperre an.
\end{itemize}

Erfindungsgemäß wird bei der Einspeisung von Referenzdaten (Replay-Daten zur Systemstabilisierung) technisch erzwungen, dass:
\begin{enumerate}
    \item Der Schreibzugriff auf neu allokierte Speicheradressen physikalisch unterbunden wird (Freeze).
    \item Lediglich die Parameter des Adress-Selektors modifiziert werden.
\end{enumerate}
Dies zwingt das System auf physikalischer Ebene, den Signalpfad für "alte" Daten von den "neuen" Speicherbereichen zu entkoppeln.

\subsection{3. Effizienzvorteil}
Durch diese Steuerung können "alte" und "neue" Datenverarbeitungsprozesse denselben physikalischen Bus und teilweise dieselben Speicherbereiche (Shared Embeddings) nutzen, ohne sich gegenseitig zu korrumpieren. Dies reduziert den Hardware-Bedarf im Vergleich zur Duplizierung um bis zu 90\%.

\newpage

% ============================================================================
% ANSPRÜCHE
% ============================================================================
\section{Patentansprüche}

\begin{enumerate}
    \item \textbf{Computerimplementiertes Verfahren} zur ressourceneffizienten Verwaltung von Speicherzugriffen in einem adaptiven Signalverarbeitungssystem, umfassend:
    \begin{itemize}
        \item Bereitstellen einer adressierbaren Datenstruktur (z.B. Matrix, Tensor) zur Speicherung von Parametern;
        \item Empfangen eines sequenziellen Eingangsdatenstroms ($x_t$);
    \end{itemize}
    \textbf{gekennzeichnet durch folgende Schritte:}
    \begin{enumerate}[label=(\alph*)]
        \item \textbf{Erzeugen eines Kontext-Steuersignals} durch Auswertung des aktuellen Eingangswerts ($x_t$) in Relation zu \textbf{mindestens einem} vorangegangenen oder parallelen Zustandswert (zeitlicher oder struktureller Kontext), um eine Signatur des aktuellen Verarbeitungsmodus zu generieren (Vermeidung der Beschränkung auf reine Bigramme);
        \item \textbf{Dynamisches Routing} des Datenstroms auf definierte Segmente der Datenstruktur in Abhängigkeit von diesem Kontext-Steuersignal;
        \item \textbf{Anwenden einer Modifikations-Restriktion} während einer Adaptionsphase, wobei in Abhängigkeit vom Kontext-Steuersignal die Änderung von Werten in definierten Speichersegmenten (z.B. Kernbereiche vs. Erweiterungsbereiche) \textbf{reduziert, skaliert oder vollständig blockiert} wird.
    \end{enumerate}
    
    \item Verfahren nach Anspruch 1, \textbf{dadurch gekennzeichnet, dass} das Kontext-Steuersignal aus einem zeitlichen Fenster der Größe $N \ge 1$ (Historie) oder aus einem Aufmerksamkeitsmechanismus (Attention) abgeleitet wird, wodurch auch komplexe Sequenzabhängigkeiten zur Trennung von Betriebsmodi genutzt werden können.
    
    \item Verfahren nach Anspruch 1 oder 2, \textbf{dadurch gekennzeichnet, dass} die Modifikations-Restriktion durch eine mathematische Operation im Rückkopplungspfad (z.B. Multiplikation mit einer Maske $M \in [0, 1]$) realisiert wird, wobei die Maskenwerte dynamisch oder statisch festgelegt sein können (Abdeckung von Soft-Freezing und Hard-Freezing).
    
    \item Verfahren nach einem der vorstehenden Ansprüche, \textbf{dadurch gekennzeichnet, dass} neu allokierte Speichersegmente initial mit Werten belegt werden, deren Betrag unterhalb eines definierten Rausch-Schwellenwerts liegt (z.B. $\approx 0$ oder $\epsilon$-Umgebung), um eine Interferenzfreiheit bei der initialen Integration zu gewährleisten (Vermeidung der Beschränkung auf exakte Null).
    
    \item \textbf{Signalverarbeitungsvorrichtung}, umfassend eine Recheneinheit und eine Speicherstruktur, konfiguriert zur Ausführung des Verfahrens nach einem der Ansprüche 1 bis 4, wobei die Vorrichtung insbesondere als Hardware-Beschleuniger (TPU, GPU, FPGA) oder eingebettetes System ausgeführt ist.
\end{enumerate}

\newpage
% ============================================================================
% ZEICHNUNGSERLÄUTERUNGEN
% ============================================================================
\section{Kurze Beschreibung der Zeichnungen}

\begin{description}
    \item[Abb. 1 (Signalfluss):] Schematische Darstellung der Kontext-Erkennung. Ein Schieberegister puffert $x_{t-1}$. Die ALU kombiniert $x_t$ und $x_{t-1}$ zu einem Adress-Signal für den Router.
    
    \item[Abb. 2 (Speicher-Layout):] Darstellung der physikalischen Speicheraufteilung in geschützte Bereiche (Core) und Erweiterungsbereiche (Sidecar). Die binäre Maske (Schreibsperre) verhindert physikalisch das Überschreiben des Cores.
\end{description}

\end{document}

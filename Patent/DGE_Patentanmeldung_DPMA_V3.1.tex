\documentclass[a4paper, 11pt]{article}

\usepackage[utf8]{inputenc}
\usepackage[T1]{fontenc}
\usepackage[ngerman]{babel}
\usepackage{geometry}
\usepackage{amsmath, amssymb}
\usepackage{fancyhdr}

\geometry{left=2.5cm, right=2.5cm, top=2.5cm, bottom=2.5cm}

\pagestyle{fancy}
\fancyhf{}
\rhead{Patentanmeldung -- Directed Synergy (DGE)}
\lhead{Sven Jansen, Dipl.-Inf.}
\cfoot{\thepage}

\section*{Zusammenfassung}
Die Erfindung betrifft eine hardware-nahe Vorrichtung und ein Verfahren zur effizienten Steuerung von Speicherressourcen in adaptiven Signalverarbeitungssystemen[cite: 1]. Ziel ist es, die funktionale Erweiterung eines Systems zu ermöglichen, ohne bestehende Datenkonfigurationen durch ungesteuerte Schreibzugriffe zu gefährden („Katastrophales Vergessen“)[cite: 2].

Die Vorrichtung umfasst einen Kontext-Detektor, der basierend auf zeitlichen Zustandsfolgen ($x_{t-1} \to x_t$) ein Auswahlsignal zur physischen Leitung des Signalflusses auf spezifische Speichersegmente erzeugt[cite: 3, 9]. Ein Kernmerkmal ist die asymmetrische Gatter-Logik, welche die Gatter-Zustände mittels einer rangreduzierten Broadcast-Addition ($g^{row} \oplus g^{col}$) berechnet[cite: 4, 10]. Dadurch wird die rechnerische Komplexität signifikant von $O(n \cdot k)$ auf $O(n + k)$ gesenkt[cite: 4]. Zusätzlich verhindert eine integrierte Rescue-Einheit das dauerhafte Einfrieren von Signalpfaden durch gezielte Signalinjektion bei Unterschreitung definierter Aktivitätsschwellenwerte[cite: 5, 11].

Die Erfindung findet Anwendung in der dynamischen Speicherpartitionierung, der Modell-Kompression sowie beim selektiven Wissenstransfer zwischen autonomen Systemen[cite: 6, 7, 8]. Durch die Trennung in einen stabilisierenden „Frozen Core“ und eine adaptive „Expansion Area“ wird eine kontrollierte Modell-Transformation bei minimalem Ressourcenverbrauch ermöglicht[cite: 7].

\begin{document}

\begin{center}
\Large\textbf{PATENTANMELDUNG}\\[0.5cm]
\large beim Deutschen Patent- und Markenamt (DPMA)\\[1cm]
\end{center}

\noindent\textbf{Bezeichnung der Erfindung:}\\
\textit{Hardware-nahe Vorrichtung und Verfahren zur dynamischen Speicherpartitionierung und Modell-Transformation in adaptiven Systemen mittels kontextabhängiger Signalgatterung}\\[0.5cm]

\section{Technisches Gebiet und Problemstellung}
Die Erfindung betrifft die effiziente Steuerung von Speicherressourcen in adaptiven Signalverarbeitungssystemen[cite: 2, 33]. Das Kernproblem besteht darin, die Funktionalität eines Systems zu erweitern oder zu transformieren, ohne die Integrität bestehender Datenkonfigurationen durch ungesteuerte Schreibzugriffe zu gefährden („Katastrophales Vergessen“)[cite: 3, 10, 39].

\section{Technische Lösung}
Die Vorrichtung löst dieses Problem durch eine Kombination aus:
\begin{enumerate}
    \item \textbf{Kontext-Routing:} Ein Detektor analysiert die Sequenzabfolge ($x_{t-1} \to x_t$), um den Signalfluss physisch auf spezifische Speichersegmente zu leiten[cite: 5, 18, 27, 48].
    \item \textbf{Rangreduzierte Gatter-Steuerung:} Die Gatter-Zustände werden über eine Broadcast-Addition ($g^{row} \oplus g^{col}$) berechnet, was die Komplexität von $O(n \cdot k)$ auf $O(n + k)$ senkt[cite: 50, 51].
    \item \textbf{Aktive Pfadreaktivierung (Rescue-Logik):} Eine Schaltung verhindert das dauerhafte Einfrieren von Signalpfaden durch gezielte Signalinjektion bei Unterschreitung von Aktivitätsschwellenwerten[cite: 53, 54].
\end{enumerate}

\section{Weitere vorteilhafte Anwendungen (Modell-Transformation)}
Die erfindungsgemäße Vorrichtung ermöglicht über das sequentielle Training hinausgehende Transformationen von Parametermatrizen:
\begin{itemize}
    \item \textbf{Modell-Kompression (Downsizing):} Durch die kontextabhängige Gatterung können redundante Signalpfade identifiziert und physisch deaktiviert werden, ohne die Performance in kritischen Betriebsmodi zu reduzieren.
    \item \textbf{Wissenstransfer (Behavior Transfer):} Die asymmetrische Schreibsteuerung erlaubt es, Verhaltensweisen eines Quell-Modells selektiv in die „Expansion Area“ eines Ziel-Modells zu projizieren [cite: 6, 22], wobei der „Frozen Core“ als stabilisierende Referenz dient[cite: 6, 35].
    \item \textbf{Model Merging:} Die Vorrichtung kann genutzt werden, um zwei unabhängig voneinander adaptierte Systeme zu verschmelzen, indem der Kontext-Router die Interferenz zwischen den jeweiligen Speichersegmenten auflöst[cite: 24, 46, 47].
\end{itemize}

\section{Patentansprüche}
\begin{enumerate}
    \item \textbf{Signalverarbeitungsvorrichtung} zur Steuerung von Speicherzugriffen, umfassend einen Kontext-Detektor zur Erzeugung eines Auswahlsignals basierend auf zeitlichen Zustandsfolgen [cite: 27, 41] und eine asymmetrische Gatter-Logik zur selektiven Schreibsperre von Speichersegmenten[cite: 4, 21].
    \item Vorrichtung nach Anspruch 1, \textbf{gekennzeichnet durch} eine Broadcast-Recheneinheit zur rangreduzierten Gatter-Steuerung[cite: 50, 51].
    \item Vorrichtung nach Anspruch 1 oder 2, \textbf{gekennzeichnet durch} eine Rescue-Einheit zur aktiven Pfadreaktivierung mittels Signalinjektion[cite: 53, 54].
    \item \textbf{Verfahren zur Modell-Transformation} unter Verwendung einer Vorrichtung nach den Ansprüchen 1 bis 3, wobei durch die Gatter-Logik ein selektiver Transfer oder eine Kompression von Parametereinstellungen durchgeführt wird.
\end{enumerate}

\end{document}